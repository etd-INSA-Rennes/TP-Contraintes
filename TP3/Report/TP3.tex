\documentclass[a4paper,12pt]{article}

\usepackage[utf8]{inputenc}
\usepackage[T1]{fontenc}
\usepackage{color}
\definecolor{grey}{rgb}{0.9,0.9,0.9}
\definecolor{teal}{rgb}{0.0,0.5,0.5}
\definecolor{violet}{rgb}{0.5,0,0.5}
\usepackage[margin=2.5cm]{geometry}
\usepackage[francais]{babel}
\usepackage{listings}
%\usepackage{graphicx}
\lstloadlanguages{Prolog}
\lstdefinestyle{listing} {
  language=Prolog,
  captionpos=t,
  inputencoding=latin1,
  extendedchars=true,
  numbers=left,
  numberstyle=\tiny,
  numbersep=5pt,
  breaklines=true,
  breakatwhitespace=true,
  showspaces=false,
  showstringspaces=false,
  showtabs=false,
  tabsize=2,
  basicstyle=\footnotesize\ttfamily,
  backgroundcolor=\color{grey},
  keywordstyle=\color{blue}\bfseries,
  commentstyle=\color{teal},
  identifierstyle=\color{black},
  stringstyle=\color{red},
  numberstyle=\color{violet},
}
\lstset{style=listing}

\title{TP3 - Ordonnancement de tâches sur deux machines}
\author{\textsc{Paul Chaignon} - \textsc{Ulysse Goarant}}
\date{\today}

\begin{document}

\maketitle

\lstinputlisting[caption=taches.ecl]{../taches.ecl}
\vspace{2cm}

\subsection*{Question 3.8}
La première solution trouvée par ECLiPSe est sûrement la meilleure parce qu'il commence le labeling avec les valeurs les plus petites. Nous devrions donc trouver en premier la solution finissant la plus tôt. Cela n'est cependant pas sûr. Nous pouvons nous en assurer en ajoutant une contrainte à la recherche de solutions (voir code 3.8).

\end{document}