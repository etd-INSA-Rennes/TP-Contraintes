\documentclass[a4paper,12pt]{article}

\usepackage[utf8]{inputenc}
\usepackage[T1]{fontenc}
\usepackage{color}
\definecolor{grey}{rgb}{0.9,0.9,0.9}
\definecolor{teal}{rgb}{0.0,0.5,0.5}
\definecolor{violet}{rgb}{0.5,0,0.5}
\usepackage[margin=2.5cm]{geometry}
\usepackage[francais]{babel}
\usepackage{listings}
%\usepackage{graphicx}
\lstloadlanguages{Prolog}
\lstdefinestyle{listing} {
  language=Prolog,
  captionpos=t,
  inputencoding=latin1,
  extendedchars=true,
  numbers=left,
  numberstyle=\tiny,
  numbersep=5pt,
  breaklines=true,
  breakatwhitespace=true,
  showspaces=false,
  showstringspaces=false,
  showtabs=false,
  tabsize=2,
  basicstyle=\footnotesize\ttfamily,
  backgroundcolor=\color{grey},
  keywordstyle=\color{blue}\bfseries,
  commentstyle=\color{teal},
  identifierstyle=\color{black},
  stringstyle=\color{red},
  numberstyle=\color{violet},
}
\lstset{style=listing}

\title{TP5 - Contraindre puis chercher}
\author{\textsc{Paul Chaignon} - \textsc{Ulysse Goarant}}
\date{\today}

\begin{document}

\maketitle

\lstinputlisting[caption=production.ecl]{../production.ecl}
\vspace{2cm}

\subsection*{Question 5.4}

Si on labelle uniquement sur $X$ le solveur nous trouve une valeur minimum de 1 ce qui est faux.
Comme les variables sont des entiers : $X = 1 \Rightarrow z = 1\ ou\ y = 1$.
Or $x \ne z + y + w$, donc ce n'est pas possible.
Le solveur a encore des \textit{delayed goals} et n'a donc pas vérifié que la valeur associée à $X$ est possible.

Il faut donc toujours labeller sur toutes les variables dans le but pour ne pas avoir de \textit{delayed goals}.


\end{document}